% chktex-file 44

\section{Parameter Descriptions}

The parameters for \textbf{DynamicFifo} are shown below in
Table~\ref{table:params}.

\renewcommand*{\arraystretch}{1.4}
\begin{longtable}[H]{
    | p{0.25\textwidth}
    | p{0.10\textwidth}
    | p{0.05\textwidth}
    | p{0.05\textwidth}
    | p{0.47\textwidth} |
  }
  \hline
  \textbf{Name} &
  \textbf{Type} &
  \textbf{Min}  &
  \textbf{Max}  &
  \textbf{Description}            \\ \hline \hline

  externalRam   &
  Boolean       &
  false         &
  true          &
  Determines whether to build FIFO memory with flip-flops or provide and
  an external interface to a SRAM \\ \hline

  dataWidth     &
  Int           &
  1             &
  $\ge$ 1       &
  The data width of the FIFO      \\ \hline

  fifoDepth     &
  Int           &
  2             &
  $\ge$ 2       &
  The depth of the FIRO           \\ \hline

  \caption{Parameter Descriptions}\label{table:params}
\end{longtable}

The DynamicFifo is instantiated into a design as follows:

\begin{lstlisting}[language=Scala]

  // Instantiate small FIFO using internal flip-flops
  val mySmallFifo = new DynamicFifo(
    externalRAM = false, 
    dataWidth = 8, 
    fifoDepth = 16) 

  // Instantiate large FIFO using external SRAM
  val myLargeFifo = new DynamicFifo(
    externalRAM = true, 
    dataWidth = 32, 
    fifoDepth = 512) 

  \end{lstlisting}