% chktex-file 44
\section{Port Descriptions}

\subsection{Timer/Counter Mode}

The ports for \textbf{Timer/Counter} are shown below in 
Table 1.
 
\renewcommand*{\arraystretch}{1.4}
\begin{longtable}[H]{
  | p{0.20\textwidth}
  | p{0.20\textwidth}
  | p{0.12\textwidth}
  | p{0.43\textwidth} |
  }
  \hline
  \textbf{Port Name} &   
  \textbf{Width} &   
  \textbf{Direction} &   
  \textbf{Description} \\ \hline \hline

  signalOut &       
  1 & 
  Output &       
  Signal generated by timer/counter\\ \hline

  irqOutput &      
  1 & 
  Output &     
  Sent when interrupt is triggered on the Gpio \\ \hline
 
 
  \caption{Timer/Counter Ports Descriptions}\label{table:tc}
\end{longtable}

\subsection{Watchdog Timer Mode}

The ports for \textbf{Watchdog Timer} are shown below in 
Table 1.
 
\renewcommand*{\arraystretch}{1.4}
\begin{longtable}[H]{
  | p{0.20\textwidth}
  | p{0.20\textwidth}
  | p{0.12\textwidth}
  | p{0.43\textwidth} |
  }
  \hline
  \textbf{Port Name} &   
  \textbf{Width} &   
  \textbf{Direction} &   
  \textbf{Description} \\ \hline \hline

  reset &       
  1 & 
  Output &       
  System reset signal\\ \hline

 
  \caption{WDT Ports Descriptions}\label{table:wdt}
\end{longtable}

\subsection{Real Time Clock Mode}

The ports for \textbf{Real Time Clock} are shown below in 
Table 1.
 
\renewcommand*{\arraystretch}{1.4}
\begin{longtable}[H]{
  | p{0.20\textwidth}
  | p{0.20\textwidth}
  | p{0.12\textwidth}
  | p{0.43\textwidth} |
  }
  \hline
  \textbf{Port Name} &   
  \textbf{Width} &   
  \textbf{Direction} &   
  \textbf{Description} \\ \hline \hline

  irqOutput &      
  1 & 
  Output &     
  Sent when interrupt is triggered on the Gpio \\ \hline

 
  \caption{RTC Ports Descriptions}\label{table:rtc}
\end{longtable}

\subsection{APB3 Interface}
The \textbf{APB3 Interface} is a regular APB3 Slave Interface. All signals supported are shown below in 
Table 2. See the \textit{AMBA APB Protocol Specifications} for a complete description of the signals. The width of several ports is controlled 
by the following input parameters:

\begin{itemize}[noitemsep]
  \item \textit{dataWidth} is the width of PWDATA and PRDATA in bits
  \item \textit{addrWidth} is the width of PADDR in bits
\end{itemize}
 
\renewcommand*{\arraystretch}{1.4}
\begin{longtable}[H]{
  | p{0.20\textwidth}
  | p{0.20\textwidth}
  | p{0.12\textwidth}
  | p{0.43\textwidth} |
  }
  \hline
  \textbf{Port Name} &   
  \textbf{Width} &   
  \textbf{Direction} &   
  \textbf{Description} \\ \hline \hline

  PCLK &       
  1 &       
  Input &       
  Positive edge clock \\ \hline

  PRESETN &       
  1 &       
  Input &       
  Active low reset \\ \hline

  PSEL &       
  1 & 
  Input &       
  Indicates slave is selected and a data transfer is required \\ \hline

  PENABLE &        
  1 & 
  Input &       
  Indicates second cycle of APB transfer \\ \hline

  PWRITE &        
  1 & 
  Input &       
  Indicates write access when HIGH and read access when LOW\\ \hline

  PADDR &      
  \textit{addrWidth} & 
  Input &     
  Address bus \\ \hline

  PWDATA &      
  \textit{dataWidth} & 
  Input &     
  Write data bus driven when PWRITE is HIGH\\ \hline

  PRDATA &      
  \textit{dataWidth} & 
  Output &     
  Read data bus driven when PWRITE is LOW\\ \hline
 
  PREADY &        
  1 & 
  Output &       
  Transfer ready \\ \hline

  PSLVERR &        
  1 & 
  Output &       
  Transfer error \\ \hline

  \caption{APB Ports Descriptions}\label{table:interface}
\end{longtable}
